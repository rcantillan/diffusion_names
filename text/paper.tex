% Added link to preamble
\documentclass[12pt]{article}
\usepackage[utf8]{inputenc}
\usepackage{blindtext}
\usepackage{graphicx}
\usepackage{authblk}
%\usepackage{natbib}
\usepackage{xcolor}
\usepackage{float}
\usepackage{amsmath}
\usepackage[english]{babel}
\usepackage{graphicx} % Required for inserting images
\usepackage[paperheight=16cm,paperwidth=12cm,textwidth=8cm]{geometry}
\usepackage[flushleft]{threeparttable,booktabs}
\usepackage{booktabs}
\usepackage{etoolbox}
\appto\TPTnoteSettings{\footnotesize}

\usepackage[
        backend=biber,
        style=authoryear-comp,
        sorting=nyt,
        style=apa
    ]{biblatex}
 \addbibresource{references.bib}


 \geometry{
 a4paper,
 total={170mm,257mm},
 left=30mm,
 right=30mm,
 top=30mm,
 bottom=30mm
 }

% Keywords command
\providecommand{\keywords}[1]
{
  \small	
  \textbf{\textit{Keywords---}} #1
}

\graphicspath{{output/}}


\addbibresource{}
\documentclass{article}


\title{Names diffusion and social class}
\author{}
\date{May 2023}

\begin{document}

\maketitle

\section{Introduction}

\section{Methodology}

We can represent the patterns by which workers change their occupations as a weighted network. In this network, each occupation is represented by a node, and each directed edge between two nodes represents the number of workers who moved from one occupation to the other. The weight of an edge can be asymmetric, meaning that the number of workers moving from one occupation to another may not be equal to the number of workers moving in the opposite direction.

The adjacency matrix of this network is simply the standard occupational mobility table. This is because each entry in the adjacency matrix, w 
ij
​
 , represents the number of workers who moved from occupation i to occupation j. Therefore, conceptualizing the mobility table as a weighted network is simply a different way of representing the same information.

Here is an example of how this might work. Suppose that we have a population of workers, and we want to track how they change occupations over time. We could start by creating a network with one node for each occupation. We could then assign a weight to each edge in the network, representing the number of workers who moved from one occupation to another. For example, if 100 workers moved from occupation A to occupation B, we would assign a weight of 100 to the edge connecting nodes A and B.

We could then track how the network changes over time by updating the weights of the edges. For example, if 50 workers moved from occupation B to occupation C, we would update the weight of the edge connecting nodes B and C to 50.

By tracking the changes in the network over time, we can get a better understanding of how workers move between occupations. This information can be used to inform policies that aim to improve labor market outcomes.

The idea of conceptualizing the mobility table as a network has several important implications. First, it allows us to think about the flow of information, influence, and other resources through the network. This is because networks are not just static structures; they are dynamic systems through which resources flow.

Second, conceptualizing the mobility table as a network allows us to go beyond direct connections to consider multistep linkages between nodes. This is because the flow of resources through a network can be indirect, as resources can flow from one node to another through intermediate nodes.

Third, conceptualizing the mobility table as a network allows us to incorporate information regarding the structure of the entire network into our analysis. This is because the structure of a network can affect the flow of resources through the network.

Here are some examples of how conceptualizing the mobility table as a network can be used to answer research questions.

We can use the concept of flow to study how information about new jobs spreads through a network. For example, we could track how quickly information about a new job opening spreads through a network of friends and colleagues.
We can use the concept of flow to study how influence spreads through a network. For example, we could track how quickly support for a new policy spreads through a network of political activists.
We can use the concept of flow to study how new ideas or innovations spread through a network. For example, we could track how quickly the use of a new technology spreads through a network of businesses.
By conceptualizing the mobility table as a network, we can gain a deeper understanding of how information, influence, and other resources flow through society. This understanding can be used to inform policies that aim to improve the flow of resources and to promote social and economic development.

\end{document}
